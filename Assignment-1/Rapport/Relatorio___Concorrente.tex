%%%%%%%%%%%%%%%%%%%%%%%%%%%%%%%%%%%%%%%%%
% Stylish Article
% LaTeX Template
% Version 2.1 (1/10/15)
%
% This template has been downloaded from:
% http://www.LaTeXTemplates.com
%
% Original author:
% Mathias Legrand (legrand.mathias@gmail.com)
% With extensive modifications by:
% Vel (vel@latextemplates.com)
%
% License:
% CC BY-NC-SA 3.0 (http://creativecommons.org/licenses/by-nc-sa/3.0/)
%
%%%%%%%%%%%%%%%%%%%%%%%%%%%%%%%%%%%%%%%%%

%----------------------------------------------------------------------------------------
%	PACKAGES AND OTHER DOCUMENT CONFIGURATIONS
%----------------------------------------------------------------------------------------

\documentclass[fleqn,10pt]{SelfArx} % Document font size and equations flushed left

\usepackage[brazil]{babel} % Specify a different language here

\usepackage{lipsum} % Required to insert dummy text. To be removed otherwise

\usepackage{adjustbox}

\usepackage{graphicx}

\usepackage{geometry}

\usepackage{float}

\usepackage{float}

\usepackage{tikz, pgf, pgfplots}

\usetikzlibrary{positioning, shadows, shapes, arrows}

\usepackage{balance}

\usepackage{cite}

\usepackage[linesnumbered,ruled]{algorithm2e}

\usepackage{nomencl}

\usepackage{microtype}

\usepackage{indentfirst}

\usepackage{comment}

\usepackage{minted}

%\usepackage{smartdiagram}
%----------------------------------------------------------------------------------------
%	COLUMNS
%----------------------------------------------------------------------------------------

\setlength{\columnsep}{0.55cm} % Distance between the two columns of text
\setlength{\fboxrule}{0.75pt} % Width of the border around the abstract

%----------------------------------------------------------------------------------------
%	COLORS
%----------------------------------------------------------------------------------------

\definecolor{color1}{RGB}{0,0,90} % Color of the article title and sections
\definecolor{color2}{RGB}{0,20,20} % Color of the boxes behind the abstract and headings


\usepackage{hyperref} % Required for hyperlinks
\hypersetup{hidelinks,colorlinks,breaklinks=true,urlcolor=color2,citecolor=color1,linkcolor=color1,bookmarksopen=false,pdftitle={Title},pdfauthor={Author}}


%%%%% Solucionates "Package PGF Math Error: Sorry, the operation 'Mod' has not yet been implemented in the floating point unit. \end{forest}" problem
\usetikzlibrary{fpu}
\def\pgfmathfloat@install@unimplemented#1{%
	\expandafter\pgfmathfloat@prepareuninstallcmd\csname pgfmath#1@\endcsname%
	\expandafter\def\csname pgfmath#1@\endcsname##1{\pgfmathfloat@notimplemented{#1}}%
}
\pgfkeys{/pgf/fpu=true}
%%%%%

%%%%% forest
%https://tex.stackexchange.com/questions/5073/making-a-simple-directory-tree
\usepackage[edges]{forest}

\definecolor{foldercolor}{RGB}{124,166,198}

\definecolor{folderbg}{RGB}{124,166,198}
\definecolor{folderborder}{RGB}{110,144,169}

\def\Size{4pt}
\tikzset{
  folder/.pic={
    \filldraw[draw=folderborder,top color=folderbg!50,bottom color=folderbg]
      (-1.05*\Size,0.2\Size+5pt) rectangle ++(.75*\Size,-0.2\Size-5pt);
    \filldraw[draw=folderborder,top color=folderbg!50,bottom color=folderbg]
      (-1.15*\Size,-\Size) rectangle (1.15*\Size,\Size);
  }
}

\forestset{is file/.style={edge path'/.expanded={%
        ([xshift=\forestregister{folder indent}]!u.parent anchor) |- (.child anchor)},
        inner sep=1pt},
    this folder size/.style={edge path'/.expanded={%
        ([xshift=\forestregister{folder indent}]!u.parent anchor) |- (.child anchor) pic[solid]{folder=#1}}, inner ysep=0.6*#1},
    folder tree indent/.style={before computing xy={l=#1}},
    folder icons/.style={folder, this folder size=#1, folder tree indent=3*#1},
    folder icons/.default={10pt},
}
%%%%%

\JournalInfo{DIM0612 - Programação Concorrente}

\Archive{Relatório Técnico - Programação com Threads}

\PaperTitle{Programação com Threads}


\Authors{Raquel Lopes de Oliveira\textsuperscript{1}} % Authors
\affiliation{\textsuperscript{1}\texttt{2013023946}}

\Keywords{Threads --- Programação --- Concorrência} % Keywords - if you don't want any simply remove all the text between the curly brackets
\newcommand{\keywordname}{Palavras-chave} % Defines the keywords heading name

%----------------------------------------------------------------------------------------
%	ABSTRACT
%----------------------------------------------------------------------------------------

\Abstract{
	 Este é um relatório desenvolvido para a disciplina de Programação Corrente, período 2017.2, professor Everton Ranielly de Sousa Cavalcante.
	 O relatório consiste na análise do registro dos tempos de execução realizando o mesmo o objetivo, multiplicação de matrizes, mas comparando a abordagem sequecial e concorrente.
}

%----------------------------------------------------------------------------------------

\begin{document}

\flushbottom % Makes all text pages the same height

\maketitle % Print the title and abstract box

%\tableofcontents % Print the contents section

\thispagestyle{empty} % Removes page numbering from the first page

%----------------------------------------------------------------------------------------
%	ARTICLE CONTENTS
%----------------------------------------------------------------------------------------
%Introdução Explicar o propósito do relatório.
\section{Introdução}
O propósito desse trabalho é realizar um teste para validar ou não a intuição em relação ao uso da concorrência. Para isso devemos fazer testes fazendo uso da concorrência e uso da solução sequencial nas mesmas condições e com diferentes parâmetros(por exemplo, número de threads).\\
No presente trabalho o algoritmo implementado foi o da multiplicação de matrizes.


%Detalhes de implementação Descrever como foi feita a sua implementação em termos de arquivos, funções/métodos, etc. e como o programa funciona de uma maneira geral.
\section{Implementação}
As implementações dos códigos foram feitas na linguagem \textit{C++}.
Para computarmos o tempo, foi usada a biblioteca chrono\footnote{http://www.cplusplus.com/reference/chrono/}. Uma classe \textit{Matrix} foi criada para representar uma matrix, uma versão dessa classe já tinha sido previamente parcialmente implementada para outra disciplina e pode ser verificada no \href{https://github.com/raquel-oliveira/numerical-analysis}{repositório numerical-analysis} no github, essa classe faz uso de template e já possui diversos métodos, que fazem jus ao nome da classe, implementados. A multiplicação do algoritmo sequencial foi feita usando a sobrecarga do operador * e para o algoritmo que faz uso de concorrência foi criado dois métodos, um responsável por fazer a atualização da matriz para determinados elementos da matriz resultante e outro(multiply) responsável por chamar o método anterior(multiplyAtomic) dado o número de threads que for ser usado.\\
O número de threads que é aceito pelo programa é no mínimo 1 e no máximo igual ao número de linhas e colunas das matrizes, dado o fato do procedimento que é realizado para a multiplicação de matrizes. Caso o número de threads seja menor que o número de linhas\footnote{ou colunas já que se trata de uma matriz quadrada}, então o número de operações atômicas(multiplicação de uma linha por uma coluna que resulta em uma célula da matriz produto) é determinado pelo seguinte cálculo:
%\setminted{fontsize=\small,baselinestretch=1}
\begin{equation*}
	nb\_op = \frac{num\_linhas \times num\_colunas}{num\_threads}
\end{equation*}
Para a última thread o cálculo tem uma alteração para que ela possua o valor do restante das operações(atômicas) que não foram realizadas:
\begin{align*}
	nb\_op = &\frac{num\_linhas \times num\_colunas}{num\_threads} +\\ &(num\_linhas \times num\_colunas)\% num\_threads
\end{align*}
Ao usar threads, usamos a chamada \textit{join()} de forma que possamos garantir que tudo já tenha sido computado antes que o programa chegue ao fim, ou seja, é a forma que temos para confirmar que todo o fluxo execução destinado aquela thread foi computado. Caso tivéssemos usado a chamada \textit{detach()} não poderíamos garantir isso, normalmente esse mecanismo é indicado para coisas que queremos que rode em \textit{background}.
\vfill
\subsection{Organização}\label{ssec:org}

\begin{forest}
	pic dir tree,
	where level=0{}{% folder icons by default; override using file for file icons
		directory,
	},
	[Concurrent-Computing
		[Assignment-1
			[data
			%[$\vdots$]
			]
			[include
				[matrix.h, file]
				[util.h, file]
			]
			[output
			%[$\vdots$]
			]
			[Rapport
			%[$\vdots$]
			]
			[time
				[concorrente]
				[sequencial]
			]
			[src
				[matrix.inl, file]
				[multimat\_concorrente.cpp, file]
				[multimat\_sequencial.cpp, file]
				[util.inl, file]
			]
			[script.sh, file]
			[Makefile, file]
			[Pratica-Threads.pdf, file]
			[Relatorio\_\_\_Concorrente, file]
		]
	]
\end{forest}\\
Os arquivos de entrada devem estar necessariamente dentro da pasta \textbf{data}, os output das matrizes produtos serão criados na pasta \textbf{output} e dentro da pasta \textbf{time} os outputs caso execute o script.sh
\subsection{Como executar}
Uma vez dentro do diretório \textit{Assignment-1} basta excutar o comando:
\begin{minted}{bash}
  $ make
\end{minted}
que será criado dois executáveis:
\textbf{multimat\_sequencial} e \textbf{multimat\_concorrente}.
Ele funciona da mesma forma como descrita na especificação do projeto:
``O programa principal deverá ser executado via linha de comando da seguinte forma:
\begin{minted}{bash}
 $./multimat_sequencial 2
\end{minted}
em que o número inteiro seguido do nome do programa representa a dimensão das matrizes quadradas que serão tratadas pelo programa. Todas as matrizes utilizadas como casos de teste para este trabalho possuem dimensões como potências de base 2, logo qualquer valor fornecido como argumento de linha de comando ao programa deve atender a essa restrição. No caso da solução concorrente, o programa principal deverá ser executado via linha de comando da seguinte forma:
\begin{minted}{bash}
$ ./multimat_concorrente 2 2
\end{minted}
em que os números inteiros seguidos do nome do programa representam, respectivamente, a dimensão
das matrizes quadradas que serão tratadas pelo programa e o número de threads a serem
utilizadas." No caso da solução concorrente, caso não seja definido o número de threads ele vai setado com o valor máximo (definido pela constante \textbf{NUMBER\_THREADS}(10) e caso a dimensão da matriz seja menor que a constante, o número de threads é igual ao valor máximo(número de colunas/linhas) dado que a a sub-tarefa mínima para uma thread é o cálculo de um elemento da matriz produto, ou seja, a multiplicação de uma linha por uma coluna.\\
Os documentos de entrada devem estar dentro da pasta \textit{data} e as matriz resultante da multiplicação se encontra na pasta \textit{output}, ambas as pastas já estão previamente criadas e é possível verificar a hierarquia na seção \ref{ssec:org}.\\

%Metodologia Indicar o método adotado para realizar os experimentos com as soluções e analisaros resultados obtidos. Por exemplo, você deve apresentar a caracterização técnica do computador utilizado (processador, sistema operacional, quantidade de memória RAM), a linguagem de programação e a versão do compilador empregados, os cenários considerados, entre outras informações. Você também deverá descrever qual o procedimento adotado para gerar os resultados, como a comparação entre as soluções foi feita, etc.
\section{Metodologia}
A máquina usada para a realização de testes foi

%Resultados Apresentar os resultados obtidos na forma de um gráficos de linha e tabelas. Para cada solução deverá ser apresentada uma tabela contendo os tempos mínimo, médio e máximo obtidos para cada dimensão de matriz. Por sua vez, os gráficos de linha deverão exibir apenas os tempos médios obtidos nas execuções de cada solução.
\newpage
\onecolumn
\section{Resultados}
Os resultados foram lançados numa planilha para realizar a devida análise, a planilha pode ser encontrada neste \href{https://docs.google.com/a/ufrn.edu.br/spreadsheets/d/1tey8rUazIv3H-rDvTVuX3HWysaB38WmbdmQCpsBp2zY/edit?usp=sharing}{link(clique aqui)}.
\begin{table}[!h]
	\centering
	\caption{Tabela - Resultados}
	\label{Tabela Resultados}
	\begin{tabular}{|c|c|cccc|}
		\hline
		\multirow{2}{*}{Dimensão} & \multirow{2}{*}{Número de Treads} & \multicolumn{4}{c}{(em milisegundos)} \\
		&                                   & Min  & Max  & Média  & Desvio padrão  \\ \hline\hline
		\multirow{4}{*}{4x4}      & 0                                &  0,41    &  0,58    &   0,45     &      0,04          \\
		& 1                                 &    0,48  &   0,94   &   0,55     &  0,12              \\
		& 2                                 &   0,53   & 1,14     & 0,59       &    0,14           \\
		& 4                                 &   0,55   &   1,75   &   0,62     &         0,25       \\ \hline
		\multirow{5}{*}{8x8}      & 0                                &      &      &        &                \\
		& 1                                 &      &      &        &                \\
		& 2                                 &      &      &        &                \\
		& 4                                 &      &      &        &                \\
		& 8                                 &      &      &        &                \\ \hline
		\multirow{6}{*}{16x16}      & 0                                &      &      &        &                \\
		& 1                                 &      &      &        &                \\
		& 2                                 &      &      &        &                \\
		& 4                                 &      &      &        &                \\
		& 8                                 &      &      &        &                \\
		& 16                                &      &      &        &                \\ \hline
		\multirow{7}{*}{32x32}      & 0                                &      &      &        &                \\
		& 1                                 &      &      &        &                \\
		& 2                                 &      &      &        &                \\
		& 4                                 &      &      &        &                \\
		& 8                                 &      &      &        &                \\
		& 16                                &      &      &        &                \\
		& 32                                &      &      &        &                \\ \hline
		\multirow{8}{*}{64x64}      & 0                                &      &      &        &                \\
		& 1                                 &      &      &        &                \\
		& 2                                 &      &      &        &                \\
		& 4                                 &      &      &        &                \\
		& 8                                 &      &      &        &                \\
		& 16                                &      &      &        &                \\
		& 32                                &      &      &        &                \\
		& 64                                &      &      &        &               \\ \hline
	\end{tabular}
\end{table}
\newpage
\begin{table}[!h]
	\centering
	\caption{Tabela - Resultados}
	\label{Tabela Resultados}
	\begin{tabular}{|c|c|cccc|}
		\hline
		\multirow{2}{*}{Dimensão} & \multirow{2}{*}{Número de Treads} & \multicolumn{4}{c}{(em milisegundos)} \\
		&                                   & Min  & Max  & Média  & Desvio padrão  \\ \hline\hline
		\multirow{8}{*}{128x128}      & 0                                &      &      &       &             \\
		& 1                                 &      &      &        &                \\
		& 2                                 &      &      &        &                \\
		& 4                                 &      &      &        &                \\
		& 8                                 &      &      &        &                \\
		& 16                                &      &      &        &                \\
		& 32                                &      &      &        &                \\
		& 64                                &      &      &        &               \\ \hline
		\multirow{8}{*}{256x256}      & 0                                &      &      &        &                \\
		& 1                                 &      &      &        &                \\
		& 2                                 &      &      &        &                \\
		& 4                                 &      &      &        &                \\
		& 8                                 &      &      &        &                \\
		& 16                                &      &      &        &                \\
		& 32                                &      &      &        &                \\
		& 64                                &      &      &        &               \\ \hline
		\multirow{8}{*}{512x512}      & 0                                &      &      &        &                \\
		& 1                                 &      &      &        &                \\
		& 2                                 &      &      &        &                \\
		& 4                                 &      &      &        &                \\
		& 8                                 &      &      &        &                \\
		& 16                                &      &      &        &                \\
		& 32                                &      &      &        &                \\
		& 64                                &      &      &        &               \\ \hline
		\multirow{8}{*}{1024x1024}      & 0                                & 7.714,75     &   17.958   &  16.488,75      &  3.516,56              \\
		& 1            &   10.934,89   & 15.776       &   14.823,35 & 964,57             \\
		& 2                                 &      &      &        &                \\
		& 4                                 &      &      &        &                \\
		& 8                                 &      &      &        &                \\
		& 16                                &      &      &        &                \\
		& 32                                &      &      &        &                \\
		& 64                                &    9.726,19  &  10.330,80    &    9.847,31    &     122,69          \\ \hline
		\multirow{8}{*}{2048x2048}      & 0                                &      &      &        &                \\
		& 1                                 &      &      &        &                \\
		& 2                                 &      &      &        &                \\
		& 4                                 &      &      &        &                \\
		& 8                                 &      &      &        &                \\
		& 16                                &      &      &        &                \\
		& 32                                &      &      &        &                \\
		& 64                                &      &      &        &               \\ \hline
	\end{tabular}
\end{table}
\twocolumn
%Discussão Discutir os resultados, ou seja, o que foi possível concluir através dos resultados obtidos através dos experimentos.
\section{Discussão}
Claramente fiz algo errado, pois os dados dos testes não me parecem corretos. \\
Mas o número de threads foi confirmado através do \textit{Monitor de atividade} do MacOS. Talvez o desempenho não melhore tanto depois de quatro threads por alguma limitação/configuração para o número de threads que podem ser executadas em paralelo. \\
Para as matrizes com dimensão menor que 256x256, eu não vi muita vantagem no uso de concorrencia. A partir da dimensão 256x256 a configuração que apresentou o melhor desempenho foi usando o parâmetro ideal para as threads igual a 4.  Ainda sim não achei a melhora tão alta quanto eu tinha idealizado antes dos testes. %No entanto, alguns diferenças na execução do sequencial comparado com a abordagem concorrente, pois o valor deveria ser similar ao sequencial ou levemente pior dado o tempo de criação da thread.

\balance
%----------------------------------------------------------------------------------------

\end{document}
