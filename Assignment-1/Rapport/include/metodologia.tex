%Metodologia Indicar o método adotado para realizar os experimentos com as soluções e analisaros resultados obtidos. Por exemplo, você deve apresentar a caracterização técnica do computador utilizado (processador, sistema operacional, quantidade de memória RAM), a linguagem de programação e a versão do compilador empregados, os cenários considerados, entre outras informações. Você também deverá descrever qual o procedimento adotado para gerar os resultados, como a comparação entre as soluções foi feita, etc.
\section{Metodologia}
A máquina usada para a realização de testes tem as seguintes características:\\
\begin{align*}
	\text{Processador}&: \text{2,7 GHz Intel Core i5}\\
	\text{Memória RAM}&: \text{16Go 1867 MHz DDR3}\\
	\text{Sistema Operacional}&: \text{macOS Sierra - version 10.12.16}\\
\end{align*}
As implementações foram feitas em C++11.\\
Compilador: g++\\
%Uso da flag -O3 para otimização e para todos os testes, foi setado a prioridade máxima para o processador.
%discussion of flag o3https://stackoverflow.com/questions/11546075/is-optimisation-level-o3-dangerous-in-g
%https://www.codingame.com/forum/t/c-and-the-o3-compilation-flag/1670
% https://stackoverflow.com/questions/19689014/gcc-difference-between-o3-and-os
%Compiling with -O3 is not a guaranteed way to improve performance, and in fact in many cases can slow down a system due to larger binaries and increased memory usage. -O3 is also known to break several packages. Therefore, using -O3 is not recommended.
